% This template uses the great work of pmichaillat who provides his 
% styles and configurations for academic papers via
%
% https://github.com/pmichaillat/latex-paper
%
% Many thanks for this! A documentation is available on
%
% https://pascalmichaillat.org/d2/
% 
% The corresponding MIT LICENCE file is part of the folder.

\documentclass[letterpaper,11pt,leqno]{article}
\usepackage{paper}
\bibliographystyle{dinat}

% Enter paper title to populate PDF metadata:
\hypersetup{pdftitle={Minimalist LaTeX Template for Academic Papers}}

% Enter path to BibTeX file with references:
\newcommand{\bib}{bibliography.bib}

% Enter path to PDF file with figures:
\newcommand{\pdf}{figures.pdf}

\begin{document}

% Enter title:
\title{Research Paper}

% Enter authors:
\author{First Author, Second Author
%
% Enter affiliations and acknowledgements:
\thanks{First Author: First University. Second Author: Second University. We thank colleagues for helpful comments and discussions. This work was supported by a grant [grant number]; another grant [grant number]; and a foundation.}}

% Enter date:
\date{Month Year}   

% Enter permanent URL (can be commented out):
% \available{https://github.com/pmichaillat/latex-paper}

\begin{titlepage}
\maketitle

% Enter abstract:
This is the abstract. Lorem ipsum dolor sit amet, consectetur adipiscing elit, sed do eiusmod tempor incididunt ut labore et dolore magna aliqua. Ut enim ad minim veniam, quis nostrud exercitation ullamco laboris nisi ut aliquip ex ea commodo consequat. Duis aute irure dolor in reprehenderit in voluptate velit esse cillum dolore eu fugiat nulla pariatur. Excepteur sint occaecat cupidatat non proident, sunt in culpa qui officia deserunt mollit anim id est laborum. Lorem ipsum dolor sit amet, consectetur adipiscing elit, sed do eiusmod tempor incididunt ut labore et dolore magna aliqua. Ut enim ad minim veniam, quis nostrud exercitation ullamco laboris nisi ut aliquip ex ea commodo consequat. Duis aute irure dolor in reprehenderit in voluptate velit esse cillum dolore eu fugiat nulla pariatur. Excepteur sint occaecat cupidatat non proident, sunt in culpa qui officia deserunt mollit anim id est laborum. Lorem ipsum dolor sit amet, consectetur adipiscing elit, sed do eiusmod tempor incididunt ut labore et dolore magna aliqua.

\end{titlepage}

\section{Introduction}\label{s:introduction}
 
Lorem ipsum dolor sit amet, consectetur adipiscing elit, sed do eiusmod tempor incididunt ut labore et dolore magna aliqua. Ut enim ad minim veniam, quis nostrud exercitation ullamco laboris nisi ut aliquip ex ea commodo consequat. Duis aute irure dolor in reprehenderit in voluptate velit esse cillum dolore eu fugiat nulla pariatur.\footnote{Excepteur sint, sunt in culpa qui officia deserunt mollit anim id est laborum.}

\subsection{Generic subsection}\label{s:section}

\subsection{Subsection with references} 

References can be appended at the end of a sentence, in parenthesis \citep{MS15}. References can also include page numbers or other bibliographical information \citep[p. 1305]{MS19}. References from the same authors and same year appear as follows: \citep{LMS18a,LMS18b}. References can be in text: for instance, \citet{M12} found this and \citet[figure 1]{M14} found that. It's also possible to collate several references to the same author: for instance, \citet{M12,M14} found things. It is also possible to cite the authors of a study by name, or the year of a study: for instance, \citeauthor{EMM21} wrote a paper in \citeyear{EMM21} on this topic.

\subsection{Subsection with paragraphs}

\paragraph{Paragraph heading} Aenean fermentum purus id lacus volutpat, a eleifend mi posuere! Mauris nec nunc commodo, vehicula enim nec; vehicula ex. Nullam euismod lorem at eros efficitur: ut ultricies ante fringilla? Nam sagittis sapien id tortor commodo---a pulvinar velit ultricies\ldots Integer ac magna velorci mollis vestibulum. Fusce id ipsum vel magna placerat vehicula. Curabitur ac lobortis justo. 

\paragraph{Another paragraph with some numbers and special characters} Pellentesque habitant 25\% morbi tristique senectus 1837--1905 et netus et malesuada fames ac turpis egestas. Integer semper euismod sapien vel dictum \#1 and \#6. Vivamus nec nunc sed metus interdum suscipit. (Maecenas tristique felis sed eleifend aliquet.) Donec et ipsum 3/4 in mauris ultricies pulvinar 9/2. Nullam quis ``sapien a justo'' vestibulum fermentum. Cras sed odio \& vitae mi placerat mollis: \$23.

\subsection{Subsection with an URL}

It is possible to insert an URL: \url{https://github.com/pmichaillat/latex-paper}. Lorem ipsum dolor sit amet, ...

\section{Section with math}\label{s:math}

This section displays a number of mathematical expressions to showcase the math fonts used in the template.

\subsection{Roman letters} 

The Roman characters in math are just the same as the characters in the text---but in \textit{italic}. Here are some small letters: 
\begin{equation*}
a\{p - r \times l\} + \frac{w(g/z+j)}{i(t)+j(t)+k(t) - e^p - x^j} + \frac{h[f]+x^f}{k[y]-e^y + x^y} = f(j)^{6+y} - i^3_{g,j,p} \approx (p_{ji})^5.
\end{equation*}
Here are some capital letters as well: 
\begin{equation*}
G[p + P^7-Q] - A_B + L\{j\} = F(X) \to [Y+K]\times Z_f/H - [g_4 - i],\;\text{for any}\;i.
\end{equation*}
The punctuation is also the same in math as in text. 

\subsection{Subsection with itemized lists}

Here is an itemized list with two levels:
\begin{itemize}
\item Et harum quidem rerum facilis est et expedita distinctio.
\item Nam libero tempore, cum soluta nobis est eligendi optio.
\item Emporibus autem quibusdam et aut officiis debitis aut rerum necessitatibus saepe eveniet. Nam libero tempore, cum soluta nobis est eligendi optio:
\begin{itemize}
\item Cumque nihil impedit
\item Quo minus id, quod maxime placeat
\item Facere possimus, omnis voluptas assumenda est
\end{itemize}
\item Et harum quidem rerum facilis est et expedita distinctio.
\item Nam libero tempore, cum soluta nobis est eligendi optio.
\end{itemize}

\section{Section with graphs}\label{s:graphs}

Here is a section with a variety of graphs. At vero eos et accusamus et iusto odio dignissimos ducimus, qui blanditiis praesentium voluptatum deleniti atque corrupti, quos dolores et quas molestias excepturi sint, obcaecati cupiditate non provident, similique sunt in culpa, qui officia deserunt mollitia animi, id est laborum et dolorum fuga. 

\section{Subsection with graphs at the top of the page}

A simple two-panel graph is on figure \ref{f:graph1}. It will be placed at the top of the page, just about here. Et harum quidem rerum facilis est et expedita distinctio. Nam libero tempore, cum soluta nobis est eligendi optio, cumque nihil impedit, quo minus id, quod maxime placeat, facere possimus, omnis voluptas assumenda est, omnis dolor repellendus. Temporibus autem quibusdam et aut officiis debitis aut rerum necessitatibus saepe eveniet, ut et voluptates repudiandae sint et molestiae non recusandae.

\begin{figure}[t]
\subcaptionbox{A first panel, 1951--2019\label{f:panel1}}{\includegraphics[scale=0.2,page=1]{\pdf}}\hfill
\subcaptionbox{A second panel, 1951--2019\label{f:panel2}}{\includegraphics[scale=0.2,page=2]{\pdf}}
\caption{Graph with two panels}
\note{This is a note for the graph. Nam libero tempore, cum soluta nobis est eligendi optio, cumque nihil impedit, quo minus id, quod maxime placeat, facere possimus.}
\label{f:graph1}\end{figure}

\section{A subsection with a simple table}

Table \ref{t:table1} is a simple table with one panel. Temporibus autem quibusdam et aut officiis debitis aut rerum necessitatibus saepe eveniet, ut et voluptates repudiandae sint et molestiae non recusandae. Itaque earum rerum hic tenetur a sapiente delectus, ut aut reiciendis voluptatibus maiores alias consequatur aut perferendis doloribus asperiores repellat. Itaque earum rerum hic tenetur a sapiente delectus, ut aut reiciendis voluptatibus maiores alias consequatur aut perferendis doloribus asperiores repellat. 

\begin{table}[t]
\caption{Basic table with one panel and multicolumns}
\begin{tabular*}{\textwidth}[]{p{4cm}@{\extracolsep\fill}cccc}
\toprule
& \multicolumn{2}{c}{Columns 1–2} & \multicolumn{2}{c}{Columns 3–4}\\
\cmidrule{2-3}\cmidrule{4-5}
& Column 1 &  Column 2 &  Column 3  &  Column 4 \\
\midrule
Line 1: & $\alpha$& $\beta$& $\gamma$ & $\delta$\\
Line 2: & $\epsilon$& $\phi$ & $\kappa$ & $\eta$  \\
Line 3: & $\kappa$& $\nu$& $\pi$ & $\kappa$ \\
Line 4: & $\psi$& $\mu$& $\nu$ & $\zeta$ \\
\bottomrule
\end{tabular*}
\note{This is a note for the table. Temporibus autem quibusdam et aut officiis debitis aut rerum necessitatibus saepe eveniet, ut et voluptates repudiandae sint et molestiae non recusandae. Pellentesque nec justo aliquet, commodo nulla sed, fringilla odio. Nullam non hendrerit nisi. Curabitur et metus vel velit blandit pharetra. Morbi interdum metus a erat bibendum, nec hendrerit eros ultricies. Vestibulum vel arcu id nulla ultricies commodo. Suspendisse potenti.}
\label{t:table1}\end{table}


\section{Conclusion}\label{s:ccl}

\paragraph{Summary}  At vero eos et accusamus et iusto odio dignissimos ducimus, qui blanditiis praesentium voluptatum deleniti atque corrupti, quos dolores et quas molestias excepturi sint, obcaecati cupiditate non provident, similique sunt in culpa, qui officia deserunt mollitia animi, id est laborum et dolorum fuga. At vero eos et accusamus et iusto odio dignissimos ducimus, qui blanditiis praesentium voluptatum deleniti atque corrupti, quos dolores et quas molestias excepturi sint.

\paragraph{Implications} ...

\bibliography{\bib}

\end{document}
